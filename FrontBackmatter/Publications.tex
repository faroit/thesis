%*******************************************************
% Publications
%*******************************************************
\pdfbookmark[1]{Publications}{publications}
\chapter*{Publications}
\label{cha:publication}

Parts of chapters 4-8 of this thesis are based on contributions that I published as first author during my time as a doctoral student.

\newcommand*{\boldnames}{}

\newbibmacro*{name:bold}[2]{%
  \def\do##1{\ifstrequal{#1, #2}{##1}{\bfseries\listbreak}{}}%
  \dolistloop{\boldnames}}


\xpretobibmacro{name:last}{\begingroup\usebibmacro{name:bold}{#1}{#2}}{}{}
\xpretobibmacro{name:first-last}{\begingroup\usebibmacro{name:bold}{#1}{#2}}{}{}
\xpretobibmacro{name:last-first}{\begingroup\usebibmacro{name:bold}{#1}{#2}}{}{}
\xpretobibmacro{name:delim}{\begingroup\normalfont}{}{}
\xapptobibmacro{name:last}{\endgroup}{}{}
\xapptobibmacro{name:first-last}{\endgroup}{}{}
\xapptobibmacro{name:last-first}{\endgroup}{}{}
\xapptobibmacro{name:delim}{\endgroup}{}{}

% \DeclareNameAlias{default}{last-first/first-last}

\DeclareFieldFormat{labelnumberwidth}{#1\adddot}
\newlength{\periodwidth}
\settowidth{\periodwidth}{.}

\defbibenvironment{numbered+bold}
  {\list
     {\printtext[labelnumberwidth]{%
        \printfield{prefixnumber}%
        \printfield{labelnumber}%
        }%
     }%
  {
   \setlength{\labelwidth}{\labelnumberwidth}%
   \setlength{\leftmargin}{\labelwidth}%
   \setlength{\labelsep}{\biblabelsep}%
   \addtolength{\labelsep}{1em}
   \addtolength{\leftmargin}{\labelsep}%
   \setlength{\itemsep}{\bibitemsep}%
   \setlength{\parsep}{\bibparsep}}%
   \renewcommand*{\makelabel}[1]{\hss##1}%
  }
  {\endlist}
  {\item\hskip-\periodwidth}


\newrefcontext[sorting=ynt]
\section*{Main Publications}

The following lists my publications that were partly included in verbatim, ordered by first appearance in the chapters of this thesis.

\subsection*{Chapter~\ref{cha:datasets}}

\begin{itemize}
  \item[\cite{stoeter15acm}] ~\fullcite{stoeter15acm}.
\end{itemize}
\noindent
Besides the leading authoring of the text, my contribution in this work was the initial idea which was inspired by the gap in existing $F_0$ estimation datasets not providing sufficient level of annotation to derive an accurate ground truth.
The work was done together with our student, Michael Müller, who much helped to design and manufacture the custom experiment hardware, organize the actual recording and provide assistance in analyzing and converting the recorded data. Bernd Edler revised the article.

\subsection*{Chapter~\ref{cha:known}}

\begin{itemize}
  \item[\cite{stoeter14}] ~\fullcite{stoeter14}.
\end{itemize}
\noindent
This work is based on my initial idea. Besides being the principal author of the text, I created the experimental designs the software implementation and evaluation.
My college Stefan Bayer contributed important insights about the theory and implementation of time warping framework and formulated the mathematical notation therein. Bernd Edler revised the article.

\begin{itemize}
  \item[\cite{stoeter15icassp}] ~\fullcite{stoeter15icassp}.
\end{itemize}
\noindent
My contribution to this work was the initial idea, the literature overview of $F_0$ estimation algorithm and the evaluation of the algorithms. 
Furthermore, I authored the main part of the text.
The work was done in close collaboration with my colleague Nils Werner who contributed to the efficient implementation of the $F_0$ warping algorithm and the generation of appropriate warp contours to match mathematical constraints of time-warping. Bernd Edler revised this publication.

\subsection*{Chapter~\ref{cha:unknown}}

\begin{itemize}
  \item[\cite{stoeter16}] ~\fullcite{stoeter16}.
\end{itemize}
\noindent
My contribution to this work was the experimental design, implementation and evaluation and the writings of the main parts of the publication.
The original idea was developed by Antoine Liutkus, who also helped to formulate the theory. Paul Magron provided code and results to compare with the HR-NMF method. Roland Badeau and Bernd Edler revised the article.

\subsection*{Chapter~\ref{cha:countanalysis}}

\begin{itemize}
  \item[\cite{stoeter13}] ~\fullcite{stoeter13}.
\end{itemize}
\noindent
The work is based on a collaboration with Michael Schöffler and Jürgen Herre.
My contribution to this work was the initial idea, as well as the experimental prototype design, and evaluation and the leading authoring of the text.
My college Michael Schöffler contributed to the development of the web-based evaluation software that later led to a follow-up publication~\cite{schoeffler13} which I co-authored. Jürgen Herre and Bernd Edler revised the article.

\subsection*{Chapter~\ref{cha:countnet}}

\begin{itemize}
  \item[\cite{stoeter19}] ~\fullcite{stoeter19}.
  \item[\cite{stoeter18}] ~\fullcite{stoeter18}.
\end{itemize}
\noindent
The publications were a result of collaboration with Soumitro Chakrabarty and Emanuël Habets. 
My contribution to this work was the initial problem formulation and the core idea to address the problem using deep neural networks. Besides the leading authoring of the text, I designed the dataset and created experiments and evaluation. 
My college Soumitro Chakrabarty contributed to the development of the deep learning method; Emanuël A. P. Habets and Bernd Edler revised the articles.


\section*{Additional Publications}
The following publications that I co-authored, were not directly referred to in this thesis but are nonetheless very closely related to audio based methods presented in this thesis. 
\begin{refsection}[ownsideref.bib]
\nocite{*}
\printbibliography[env=numbered+bold, heading=none,resetnumbers=true, sorting=ynt]
\newrefcontext[sorting=nyt]
\end{refsection}

\section*{Open Datasets and Software}
\label{sec:opendatasets}
To foster reproducible research, the following datasets and code were contributed under open licenses:
\begin{refsection}[owndata.bib]
\nocite{*}
\printbibliography[env=numbered+bold, heading=none,resetnumbers=true, sorting=ynt]
\newrefcontext[sorting=nyt]
\end{refsection}
