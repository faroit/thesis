%*******************************************************
% Abstract
%*******************************************************
%\renewcommand{\abstractname}{Abstract}
\pdfbookmark[1]{Abstract}{Abstract}
% \addcontentsline{toc}{chapter}{\tocEntry{Abstract}}
\begingroup
\let\clearpage\relax
\let\cleardoublepage\relax
\let\cleardoublepage\relax

\chapter*{Abstract}
Everyday audio recordings involve mixture signals: music contains a mixture of instruments; in a meeting or conference, there is a mixture of human voices. 
For these mixtures, automatically separating or estimating the number of sources is a challenging task.
A common assumption when processing mixtures in the time-frequency domain is that sources are not fully overlapped.
However, in this work we consider some cases where the overlap is severe --- for instance, when instruments play the same note (unison) or when many people speak concurrently ("cocktail party") --- highlighting the need for new representations and more powerful models.
\par
To address the problems of source separation and count estimation, we use conventional signal processing techniques as well as deep neural networks (DNN).
We first address the source separation problem for unison instrument mixtures, studying the distinct spectro-temporal modulations caused by vibrato.
To exploit these modulations, we developed a method based on time warping, informed by an estimate of the fundamental frequency.
For cases where such estimates are not available, we present an unsupervised model, inspired by the way humans group time-varying sources (common fate).
This contribution comes with a novel representation that improves separation for overlapped and modulated sources on unison mixtures but also improves vocal and accompaniment separation when used as an input for a DNN model.
\par
Then, we focus on estimating the number of sources in a mixture, which is important for real-world scenarios.
Our work on count estimation was motivated by a study on how humans can address this task, which lead us to conduct listening experiments, confirming that humans are only able to estimate the number of up to four sources correctly.
To answer the question of whether machines can perform similarly, we present a DNN architecture, trained to estimate the number of concurrent speakers.
Our results show improvements compared to other methods, and the model even outperformed humans on the same task.
\par
In both the source separation and source count estimation tasks, the key contribution of this thesis is the concept of ``modulation'', which is important to computationally mimic human performance. 
Our proposed common fate transform is an adequate representation to disentangle overlapping signals for separation, and an inspection of our DNN count model revealed that it proceeds to find modulation-like intermediate features.
\vfill

\begin{otherlanguage}{ngerman}
\pdfbookmark[1]{Zusammenfassung}{Zusammenfassung}
\chapter*{Zusammenfassung}

In unserem Alltag sind wir ständig von gemischten Signalen umgeben: Musik besteht aus einer Mischung von Instrumenten; in einem Meeting oder auf einer Konferenz sind wir einer Mischung menschlicher Stimmen ausgesetzt.
Obwohl wir Menschen unsere Aufmerksamkeit auf bestimmte Klangquellen richten können, bleibt die automatische Zählung und Trennung der Quellen eine anspruchsvolle Aufgabe für den Computer.
Eine häufige Annahme bei der Verarbeitung von gemischten Signalen im Zeit-Frequenzbereich ist, dass ihre Quellen nicht vollständig überlappend sind.
In dieser Arbeit betrachten wir jedoch einige Fälle, in denen die Überlappung immens ist --- zum Beispiel, wenn Instrumente den gleichen Ton spielen (unisono) oder wenn viele Menschen gleichzeitig sprechen (Cocktailparty) --- ,so dass die Notwendigkeit nach neuen Repräsentationen und leistungsfähigeren Modelle besteht.
Um die zwei genanten Probleme zu bewältigen, haben wir sowohl konventionelle Signalverbeitungsmethoden als  auch tiefgehende neuronale Netze (DNN) verwendet.
Wir gehen zunächst auf das Problem der Quellentrennung für unisono Instrumentenmischungen ein und untersuchen die speziellen, zeitlich-spektralen Modulationen, ausgelöst durch das Vibrato.
Um diese Modulationen auszunutzen, entwickelten wir eine Methode, die auf dem Zeitverzerrung basiert und eine Schätzung der Grundfrequenz als zusätzliche Information nutzt.

Wenn zwei Instrumente den gleichen Ton spielen (unisono) oder wenn viele Menschen gleichzeitig sprechen (``Cocktail-Party''), ist die Überlappung groß, was die Notwendigkeit an neuen Repräsentationen und leistungsfähiger Modelle zur Bewältigung dieser beiden Aufgaben unterstreicht.
Bei der Trennung von Quellen im Zeit-Frequenzbereich (wie bei der nicht-negativen Matrixfaktorisierung) wird häufig angenommen, dass die Quellen nicht vollständig überlappt sind.
Die meisten Ansätze haben sich jedoch auf nicht überlappende Segmente konzentriert um eine Zählung durch Erkennung zu erleichtern, aber mit einer grösseren Überlappung müssen die Strategien hinzu eine direkten  Anzahl verlagert werden.
Die genaue Schätzung der Anzahl von Quellen ist für reale Szenarien sehr wichtig.
Um beide Probleme zu lösen, verwendeten wir sowohl konventionelle Signalverarbeitungstechniken als auch tiefgehendes neuronale Netze (DNN).
Wir gehen zunächst auf das Problem der Quellentrennung für unisono Instrumentenmischungen ein und untersuchen gründlich die unterschiedlichen zeitlich-spektralen Modulationen, verursacht durch Vibrato. 
Um diese Modulationen auszunutzen, entwickeln wir eine Methode, die auf dem Zeitverzerrung basiert und eine Schätzung der Grundfrequenz als zusätzliche Information nutzt.
Für Fälle, in denen diese Schätzungen nicht verfügbar sind, stellen wir ein unüberwachtes Modell vor, das inspiriert ist von der Art und Weise  wie Menschen zeitveränderliche Quellen gruppieren (Common Fate).
Dieser Beitrag enthält eine neuartige Repräsentation, die die Separierbarkeit für überlappte und modulierte Quellen in  Unisono-Mischungen erhöht, aber auch die Trennung in Gesang und Begleitung verbessert, wenn sie eingangs für ein DNN-Modell verwendet wird.
Wir bearbeiten die Aufgabe, die Anzahl von Quellen zu schätzen, indem wir zunächst untersuchen wie Menschen diese Aufgaben durch Hörversuche lösen können und bestätigen, dass Menschen nur in der Lage sind, bis zu vier Quellen korrekt in der Anzahl zu schätzen.
Um die Frage zu beantworten, ob Maschinen diese Aufgabe ähnlich gut bewältigen können, stellen wir eine DNN-Architektur vor, die erlernt hat, die Anzahl der gleichzeitigen Sprecher in einer Cocktail-Party-Umgebung von bis zu zehn Sprechern zu schätzen.
Unsere Ergebnisse zeigen Verbesserungen im Vergleich zu anderen Methoden, und das Modell übertraf sogar die Leistung des Menschen bei in der gleichen Aufgabe.
In dieser Arbeit haben wir bestätigt, wie wichtig die modulationsbasierte Signaltrennung ist. 
Unser Modell zur Schätzung der Sprecherzahl ermöglicht Anwendungen wie die Überwachung von Menschenmassen oder die beantwortung der Frage nach ``Wer spricht wann?'' in schwierigen Umgebungen.
Schließlich haben wir unser Schätzmodell für die Anzahl genau untersucht, um Erkenntnisse darüber zu gewinnen, warum es so leistungsfähig ist, und festgestellt, dass Modulationen auch hier eine entscheidende Rolle spielen.
\end{otherlanguage}

\endgroup

\vfill
