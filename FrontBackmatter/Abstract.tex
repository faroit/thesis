%*******************************************************
% Abstract
%*******************************************************
%\renewcommand{\abstractname}{Abstract}
\pdfbookmark[1]{Abstract}{Abstract}
% \addcontentsline{toc}{chapter}{\tocEntry{Abstract}}
\begingroup
\let\clearpage\relax
\let\cleardoublepage\relax
\let\cleardoublepage\relax

\chapter*{Abstract}
%%%%% What is the problem?
% Not more than 3-4 sentences telling the reader what the problem is, in as simple English as possible
Audio signals that overlap in time and frequency are a common source of difficulty for many tasks that process or analyze these mixtures.
To separate sources in the time-frequency domain, it is assumed that they are not fully overlapped and methods exploit this aspect in techniques such as non-negative matrix factorization.
Although most scientific efforts have focused on separation, we also consider the task of estimating the number of sources where non-overlapping segments can be used to facilitate ``count by detection''.
However, when two instruments play the same note (unison) or when many talkers speak at the same time (``cocktail party''), the overlap is severe, highlighting the need for new representations and more powerful models in order to address these two tasks.
%%%%% Why is the problem hard? 
% What has eluded us in solving it?
% What does the literature say about this problem?
% What are the obstacles/challenges? Why is it non-trivial? 
Existing separation methods often perform poorly because underlying time-frequency representations do not offer a sufficient level of separability.
Source count estimation is a hard and abstract problem when sources are overlapped severely and strategies need to shift towards directly inferring a count.
%%%%% What is your approach/result to solving this problem?
% How come you solved it? 
% sit up and take notice claims that your thesis will plan to prove/demonstrate 
To address both problems, we used conventional signal processing techniques as well as the recent deep learning framework.
More specifically, we first address the source separation problem for unison instruments mixtures.
Here, we discovered the distinct tempo-spectral modulations caused by vibrato. 
In a first step to exploit these modulations, we develop a method based on time warping that relies on an estimate of the fundamental frequency variation of the sources. 
For cases where such estimates are not available, we present a model based on non-negative tensor factorization and inspired by the way humans group time-varying sources (common fate).
This contribution comes with a novel representation that improves the separability for overlapped and modulated sources.
We developed and present datasets to evaluate these models on unison mixtures.
Furthermore, we showed that this representation can be used as an input for a deep neural network based model to also improve vocal/accompaniment separation.
\par
We address the count estimation task by first study how humans can solve this task. 
Therefore, we conducted listening experiments for both, overlapped speech as well as polyphonic music recordings. 
These experiments confirm that humans could only count up to four sources.
In order to answer the question if machines can perform similarly, we present CountNet, a DNN architecture, trained to estimate the maximum number of concurrent speakers in a cocktail party scenario of up to ten speakers.
Our results show significant improvements compared to other methods. 
Furthermore, the model outperformed humans on the same task.
\par
%%%%% What is the consequence of your approach?
% So, now that you’ve made me sit up and take notice, what is the impact? 
% What does your approach/result enable?
Modulation based based processing
An accurate estimate of the number of sources is of paramount importance to address any real-world source separation scenario.
Our model enables application such as crowd counting or to further address speaker diarization (who speaks when) in meeting rooms.


\vfill

\begin{otherlanguage}{ngerman}
\pdfbookmark[1]{Zusammenfassung}{Zusammenfassung}
\chapter*{Zusammenfassung}
t.b.a\dots
\end{otherlanguage}

\endgroup

\vfill
