% ****************************************************************************************************
% classicthesis-config.tex
% formerly known as loadpackages.sty, classicthesis-ldpkg.sty, and classicthesis-preamble.sty
% Use it at the beginning of your ClassicThesis.tex, or as a LaTeX Preamble
% in your ClassicThesis.{tex,lyx} with % ****************************************************************************************************
% classicthesis-config.tex
% formerly known as loadpackages.sty, classicthesis-ldpkg.sty, and classicthesis-preamble.sty
% Use it at the beginning of your ClassicThesis.tex, or as a LaTeX Preamble
% in your ClassicThesis.{tex,lyx} with % ****************************************************************************************************
% classicthesis-config.tex
% formerly known as loadpackages.sty, classicthesis-ldpkg.sty, and classicthesis-preamble.sty
% Use it at the beginning of your ClassicThesis.tex, or as a LaTeX Preamble
% in your ClassicThesis.{tex,lyx} with % ****************************************************************************************************
% classicthesis-config.tex
% formerly known as loadpackages.sty, classicthesis-ldpkg.sty, and classicthesis-preamble.sty
% Use it at the beginning of your ClassicThesis.tex, or as a LaTeX Preamble
% in your ClassicThesis.{tex,lyx} with \input{classicthesis-config}
% ****************************************************************************************************
% If you like the classicthesis, then I would appreciate a postcard.
% My address can be found in the file ClassicThesis.pdf. A collection
% of the postcards I received so far is available online at
% http://postcards.miede.de
% ****************************************************************************************************


% ****************************************************************************************************
% 0. Set the encoding of your files. UTF-8 is the only sensible encoding nowadays. If you can't read
% äöüßáéçèê∂åëæƒÏ€ then change the encoding setting in your editor, not the line below. If your editor
% does not support utf8 use another editor!
% ****************************************************************************************************
\PassOptionsToPackage{utf8}{inputenc}
  \usepackage{inputenc}

\PassOptionsToPackage{T1}{fontenc} % T2A for cyrillics
  \usepackage{fontenc}


% ****************************************************************************************************
% 1. Configure classicthesis for your needs here, e.g., remove "drafting" below
% in order to deactivate the time-stamp on the pages
% (see ClassicThesis.pdf for more information):
% ****************************************************************************************************
\PassOptionsToPackage{
  drafting=true,    % print version information on the bottom of the pages
  tocaligned=false, % the left column of the toc will be aligned (no indentation)
  dottedtoc=false,  % page numbers in ToC flushed right
  eulerchapternumbers=true, % use AMS Euler for chapter font (otherwise Palatino)
  linedheaders=false,       % chaper headers will have line above and beneath
  floatperchapter=true,     % numbering per chapter for all floats (i.e., Figure 1.1)
  eulermath=false,  % use awesome Euler fonts for mathematical formulae (only with pdfLaTeX)
  beramono=true,    % toggle a nice monospaced font (w/ bold)
  palatino=true,    % deactivate standard font for loading another one, see the last section at the end of this file for suggestions
  style=arsclassica % classicthesis, arsclassica
}{classicthesis}


% ****************************************************************************************************
% 2. Personal data and user ad-hoc commands (insert your own data here)
% ****************************************************************************************************
\input{revision}

% Separation and Count Estimation for Audio Signals Overlapping in Time and Frequency
% Trennung und Schätzung der Anzahl von Zeit- und Frequenzüberlappenden Audiosignalen

\newcommand{\myTitle}{Separation and Count Estimation for Audio Sources Overlapping in Time and Frequency\xspace}
\newcommand{\myTitleGerman}{Trennung und Schätzung der Anzahl von Audiosignalquellen mit Zeit- und Frequenzüberlappung\xspace}
\newcommand{\mySubtitle}{test\xspace}
\newcommand{\myDegree}{Doktor-Ingenieur (Dr.-Ing.)\xspace}
\newcommand{\myName}{Fabian-Robert Stöter\xspace}
\newcommand{\myProf}{Prof. Dr.-Ing. Bernd Edler\xspace}
\newcommand{\myOtherProf}{Put name here\xspace}
\newcommand{\mySupervisor}{Put name here\xspace}
\newcommand{\myFaculty}{International Audio Laboratories Erlangen\xspace}
\newcommand{\myDepartment}{\xspace}
\newcommand{\myUni}{Universität Erlangen-Nürnberg\xspace}
\newcommand{\myLocation}{Montpellier\xspace}
\newcommand{\myTime}{22. November 2018\xspace}
\newcommand{\myVersion}{commit \revision}

% ********************************************************************
% Setup, finetuning, and useful commands
% ********************************************************************
\providecommand{\mLyX}{L\kern-.1667em\lower.25em\hbox{Y}\kern-.125emX\@}
\newcommand{\ie}{i.\,e.}
\newcommand{\Ie}{I.\,e.}
\newcommand{\eg}{e.\,g.}
\newcommand{\Eg}{E.\,g.}
% ****************************************************************************************************


% ****************************************************************************************************
% 3. Loading some handy packages
% ****************************************************************************************************
% ********************************************************************
% Packages with options that might require adjustments
% ********************************************************************
\PassOptionsToPackage{ngerman,american}{babel} % change this to your language(s), main language last
% Spanish languages need extra options in order to work with this template
%\PassOptionsToPackage{spanish,es-lcroman}{babel}
    \usepackage{babel}

\usepackage{csquotes}
\PassOptionsToPackage{%
  backend=biber,bibencoding=utf8, %instead of bibtex
  % backend=bibtex8,bibencoding=ascii,%
  language=auto,%
  % style=ieee,%
  % dashed=false,
  defernumbers,
  style=alphabetic, % Author 1999, 2010
  %bibstyle=authoryear,dashed=false, % dashed: substitute rep. author with ---
  sorting=nyt, % name, year, title
  maxbibnames=10, % default: 3, et al.
  %backref=true,%
  natbib=false % natbib compatibility mode (\citep and \citet still work)
}{biblatex}

\usepackage{biblatex}

\ExecuteBibliographyOptions{%
  sorting=nty, % Sort by name, title, year.
  maxbibnames=15,%
}%

% Emphasize own name in References with boldface
\usepackage{xpatch}

\usepackage{amsmath}

\usepackage{float}
\usepackage{amsmath,amssymb,amsfonts,amsthm}
\usepackage{euscript}

%----------------------
\usepackage{tikz}
\usetikzlibrary{positioning,fit, calc}
\usepackage{pgfplots}

\usepackage{bm}

% ********************************************************************
% General useful packages
% ********************************************************************
\usepackage{graphicx} %
\usepackage{float}
\usepackage{scrhack} % fix warnings when using KOMA with listings package
\usepackage{xspace} % to get the spacing after macros right
\PassOptionsToPackage{printonlyused,smaller}{acronym}
  \usepackage{acronym} % nice macros for handling all acronyms in the thesis
  %\renewcommand{\bflabel}[1]{{#1}\hfill} % fix the list of acronyms --> no longer working
  %\renewcommand*{\acsfont}[1]{\textsc{#1}}
  %\renewcommand*{\aclabelfont}[1]{\acsfont{#1}}
  %\def\bflabel#1{{#1\hfill}}
  \def\bflabel#1{{\acsfont{#1}\hfill}}
  \def\aclabelfont#1{\acsfont{#1}}
% ****************************************************************************************************
\usepackage{pgfplots} % External TikZ/PGF support (thanks to Andreas Nautsch)
% \usetikzlibrary{external}
% \tikzexternalize[mode=list and make, prefix=ext-tikz/]
% ****************************************************************************************************


% ****************************************************************************************************
% 4. Setup floats: tables, (sub)figures, and captions
% ****************************************************************************************************
\usepackage{tabularx} % better tables
  \setlength{\extrarowheight}{3pt} % increase table row height
\newcommand{\tableheadline}[1]{\multicolumn{1}{l}{\spacedlowsmallcaps{#1}}}
\newcommand{\myfloatalign}{\centering} % to be used with each float for alignment

% ****************************************************************************************************

% ****************************************************************************************************
% 5. Setup code listings
% ****************************************************************************************************
% \usepackage{listings}
\usepackage{longtable}

%\lstset{emph={trueIndex,root},emphstyle=\color{BlueViolet}}%\underbar} % for special keywords
% \lstset{language=[LaTeX]Tex,%C++,
%   morekeywords={PassOptionsToPackage,selectlanguage},
%   keywordstyle=\color{RoyalBlue},%\bfseries,
%   basicstyle=\small\ttfamily,
%   %identifierstyle=\color{NavyBlue},
%   commentstyle=\color{Green}\ttfamily,
%   stringstyle=\rmfamily,
%   numbers=none,%left,%
%   numberstyle=\scriptsize,%\tiny
%   stepnumber=5,
%   numbersep=8pt,
%   showstringspaces=false,
%   breaklines=true,
%   %frameround=ftff,
%   %frame=single,
%   belowcaptionskip=.75\baselineskip
%   %frame=L
% }
% ****************************************************************************************************
\usepackage{setspace}                                                                                               
\usepackage{minted} % Syntax highlighting                                                                                                      
\usemintedstyle{tango}
\definecolor{tango-bg}{HTML}{F8F8F8}

\newminted{python}{bgcolor=tango-bg,frame=lines,framesep=2mm,samepage=true,fontsize=\footnotesize}


% ****************************************************************************************************
% 6. Last calls before the bar closes
% ****************************************************************************************************
% ********************************************************************
% Her Majesty herself
% ********************************************************************
\usepackage{classicthesis}


% ********************************************************************
% Fine-tune hyperreferences (hyperref should be called last)
% ********************************************************************
\hypersetup{%
  %draft, % hyperref's draft mode, for printing see below
  colorlinks=true, linktocpage=true, pdfstartpage=3, pdfstartview=FitV,%
  % uncomment the following line if you want to have black links (e.g., for printing)
  %colorlinks=false, linktocpage=false, pdfstartpage=3, pdfstartview=FitV, pdfborder={0 0 0},%
  breaklinks=true, pageanchor=true,%
  pdfpagemode=UseNone, %
  % pdfpagemode=UseOutlines,%
  plainpages=false, bookmarksnumbered, bookmarksopen=true, bookmarksopenlevel=1,%
  hypertexnames=true, pdfhighlight=/O,%nesting=true,%frenchlinks,%
  urlcolor=CTurl, linkcolor=CTlink, citecolor=CTcitation, %pagecolor=RoyalBlue,%
  %urlcolor=Black, linkcolor=Black, citecolor=Black, %pagecolor=Black,%
  pdftitle={\myTitle},%
  pdfauthor={\textcopyright\ \myName, \myUni, \myFaculty},%
  pdfsubject={},%
  pdfkeywords={},%
  pdfcreator={pdfLaTeX},%
  pdfproducer={LaTeX with hyperref and classicthesis}%
}


% ********************************************************************
% Setup autoreferences (hyperref and babel)
% ********************************************************************
% There are some issues regarding autorefnames
% http://www.tex.ac.uk/cgi-bin/texfaq2html?label=latexwords
% you have to redefine the macros for the
% language you use, e.g., american, ngerman
% (as chosen when loading babel/AtBeginDocument)
% ********************************************************************
\makeatletter
\@ifpackageloaded{babel}%
  {%
    \addto\extrasamerican{%
      \renewcommand*{\figureautorefname}{Figure}%
      \renewcommand*{\tableautorefname}{Table}%
      \renewcommand*{\partautorefname}{Part}%
      \renewcommand*{\chapterautorefname}{Chapter}%
      \renewcommand*{\sectionautorefname}{Section}%
      \renewcommand*{\subsectionautorefname}{Section}%
      \renewcommand*{\subsubsectionautorefname}{Section}%
    }%
    \addto\extrasngerman{%
      \renewcommand*{\paragraphautorefname}{Absatz}%
      \renewcommand*{\subparagraphautorefname}{Unterabsatz}%
      \renewcommand*{\footnoteautorefname}{Fu\"snote}%
      \renewcommand*{\FancyVerbLineautorefname}{Zeile}%
      \renewcommand*{\theoremautorefname}{Theorem}%
      \renewcommand*{\appendixautorefname}{Anhang}%
      \renewcommand*{\equationautorefname}{Gleichung}%
      \renewcommand*{\itemautorefname}{Punkt}%
    }%
      % Fix to getting autorefs for subfigures right (thanks to Belinda Vogt for changing the definition)
      \providecommand{\subfigureautorefname}{\figureautorefname}%
    }{\relax}
\makeatother


% ********************************************************************
% Development Stuff
% ********************************************************************
\listfiles
%\PassOptionsToPackage{l2tabu,orthodox,abort}{nag}
%  \usepackage{nag}
%\PassOptionsToPackage{warning, all}{onlyamsmath}
%  \usepackage{onlyamsmath}


% ****************************************************************************************************
% 7. Further adjustments (experimental)
% ****************************************************************************************************
% ********************************************************************
% Changing the text area
% ********************************************************************
%\areaset[current]{312pt}{761pt} % 686 (factor 2.2) + 33 head + 42 head \the\footskip
%\setlength{\marginparwidth}{7em}%
%\setlength{\marginparsep}{2em}%

% ********************************************************************
% Using different fonts
% ********************************************************************
%\usepackage[oldstylenums]{kpfonts} % oldstyle notextcomp
% \usepackage[osf]{libertine}
%\usepackage[light,condensed,math]{iwona}
%\renewcommand{\sfdefault}{iwona}
%\usepackage{lmodern} % <-- no osf support :-(
%\usepackage{cfr-lm} %
%\usepackage[urw-garamond]{mathdesign} <-- no osf support :-(
%\usepackage[default,osfigures]{opensans} % scale=0.95
% \usepackage[sfdefault]{FiraSans}
% \usepackage[opticals,mathlf]{MinionPro} % onlytext
% ********************************************************************
%\usepackage[largesc,osf]{newpxtext}
%\linespread{1.05} % a bit more for Palatino
% Used to fix these:
% https://bitbucket.org/amiede/classicthesis/issues/139/italics-in-pallatino-capitals-chapter
% https://bitbucket.org/amiede/classicthesis/issues/45/problema-testatine-su-classicthesis-style
% ********************************************************************
% ****************************************************************************************************

\newcommand{\contribs}[1]{
  \begin{minipage}{.3\linewidth}
  \end{minipage}
  \hfill
  \begin{minipage}{.67\linewidth}
    %\begin{flushright}
      \begin{small}
        \textcolor{gray}{\textsf{#1}}
      \end{small}
      \vspace{20pt}\\
    %\end{flushright}
  \end{minipage}
}

\input{mathcommands.tex}

\usepackage{siunitx}

\floatstyle{ruled}
\newfloat{algorithm}{tbp}{loa}
\providecommand{\algorithmname}{Algorithm}
\floatname{algorithm}{\protect\algorithmname}

% Example definitions.
\def\x{{\mathbf{x}}}
\def\L{{\cal{L}}}
\def\Pitch{{F}}

% Official addiontional AudioLabs colors
\definecolor{nice}{rgb}{0.8,0.725490196078431,0.454901960784314}
\definecolor{almagenta}{RGB}{210,85,255}
\definecolor{alturquise}{RGB}{43,174,91}
\definecolor{algreen}{RGB}{111,217,0}
\definecolor{alturquisedark}{RGB}{0,151,164}

% Example definitions.
% --------------------
\providecommand{\e}[1]{\ensuremath{\times 10^{#1}}}


% \renewcommand\formatchapter[1]{%
%     \setbox0=\hbox{\chapterNumber\thechapter\hspace{10pt}\vline\ }%
%     \begin{minipage}[t]{\dimexpr\linewidth-\wd0\relax}%
%     \raggedright\spacedallcaps{#1}%
%     \end{minipage}%
% }

% ****************************************************************************************************
% If you like the classicthesis, then I would appreciate a postcard.
% My address can be found in the file ClassicThesis.pdf. A collection
% of the postcards I received so far is available online at
% http://postcards.miede.de
% ****************************************************************************************************


% ****************************************************************************************************
% 0. Set the encoding of your files. UTF-8 is the only sensible encoding nowadays. If you can't read
% äöüßáéçèê∂åëæƒÏ€ then change the encoding setting in your editor, not the line below. If your editor
% does not support utf8 use another editor!
% ****************************************************************************************************
\PassOptionsToPackage{utf8}{inputenc}
  \usepackage{inputenc}

\PassOptionsToPackage{T1}{fontenc} % T2A for cyrillics
  \usepackage{fontenc}


% ****************************************************************************************************
% 1. Configure classicthesis for your needs here, e.g., remove "drafting" below
% in order to deactivate the time-stamp on the pages
% (see ClassicThesis.pdf for more information):
% ****************************************************************************************************
\PassOptionsToPackage{
  drafting=true,    % print version information on the bottom of the pages
  tocaligned=false, % the left column of the toc will be aligned (no indentation)
  dottedtoc=false,  % page numbers in ToC flushed right
  eulerchapternumbers=true, % use AMS Euler for chapter font (otherwise Palatino)
  linedheaders=false,       % chaper headers will have line above and beneath
  floatperchapter=true,     % numbering per chapter for all floats (i.e., Figure 1.1)
  eulermath=false,  % use awesome Euler fonts for mathematical formulae (only with pdfLaTeX)
  beramono=true,    % toggle a nice monospaced font (w/ bold)
  palatino=true,    % deactivate standard font for loading another one, see the last section at the end of this file for suggestions
  style=arsclassica % classicthesis, arsclassica
}{classicthesis}


% ****************************************************************************************************
% 2. Personal data and user ad-hoc commands (insert your own data here)
% ****************************************************************************************************
% Autogenerated, do not edit
\newcommand{\revisiondate}{2020-01-17}
\newcommand{\revision}{5d7b1d0}


% Separation and Count Estimation for Audio Signals Overlapping in Time and Frequency
% Trennung und Schätzung der Anzahl von Zeit- und Frequenzüberlappenden Audiosignalen

\newcommand{\myTitle}{Separation and Count Estimation for Audio Sources Overlapping in Time and Frequency\xspace}
\newcommand{\myTitleGerman}{Trennung und Schätzung der Anzahl von Audiosignalquellen mit Zeit- und Frequenzüberlappung\xspace}
\newcommand{\mySubtitle}{test\xspace}
\newcommand{\myDegree}{Doktor-Ingenieur (Dr.-Ing.)\xspace}
\newcommand{\myName}{Fabian-Robert Stöter\xspace}
\newcommand{\myProf}{Prof. Dr.-Ing. Bernd Edler\xspace}
\newcommand{\myOtherProf}{Put name here\xspace}
\newcommand{\mySupervisor}{Put name here\xspace}
\newcommand{\myFaculty}{International Audio Laboratories Erlangen\xspace}
\newcommand{\myDepartment}{\xspace}
\newcommand{\myUni}{Universität Erlangen-Nürnberg\xspace}
\newcommand{\myLocation}{Montpellier\xspace}
\newcommand{\myTime}{22. November 2018\xspace}
\newcommand{\myVersion}{commit \revision}

% ********************************************************************
% Setup, finetuning, and useful commands
% ********************************************************************
\providecommand{\mLyX}{L\kern-.1667em\lower.25em\hbox{Y}\kern-.125emX\@}
\newcommand{\ie}{i.\,e.}
\newcommand{\Ie}{I.\,e.}
\newcommand{\eg}{e.\,g.}
\newcommand{\Eg}{E.\,g.}
% ****************************************************************************************************


% ****************************************************************************************************
% 3. Loading some handy packages
% ****************************************************************************************************
% ********************************************************************
% Packages with options that might require adjustments
% ********************************************************************
\PassOptionsToPackage{ngerman,american}{babel} % change this to your language(s), main language last
% Spanish languages need extra options in order to work with this template
%\PassOptionsToPackage{spanish,es-lcroman}{babel}
    \usepackage{babel}

\usepackage{csquotes}
\PassOptionsToPackage{%
  backend=biber,bibencoding=utf8, %instead of bibtex
  % backend=bibtex8,bibencoding=ascii,%
  language=auto,%
  % style=ieee,%
  % dashed=false,
  defernumbers,
  style=alphabetic, % Author 1999, 2010
  %bibstyle=authoryear,dashed=false, % dashed: substitute rep. author with ---
  sorting=nyt, % name, year, title
  maxbibnames=10, % default: 3, et al.
  %backref=true,%
  natbib=false % natbib compatibility mode (\citep and \citet still work)
}{biblatex}

\usepackage{biblatex}

\ExecuteBibliographyOptions{%
  sorting=nty, % Sort by name, title, year.
  maxbibnames=15,%
}%

% Emphasize own name in References with boldface
\usepackage{xpatch}

\usepackage{amsmath}

\usepackage{float}
\usepackage{amsmath,amssymb,amsfonts,amsthm}
\usepackage{euscript}

%----------------------
\usepackage{tikz}
\usetikzlibrary{positioning,fit, calc}
\usepackage{pgfplots}

\usepackage{bm}

% ********************************************************************
% General useful packages
% ********************************************************************
\usepackage{graphicx} %
\usepackage{float}
\usepackage{scrhack} % fix warnings when using KOMA with listings package
\usepackage{xspace} % to get the spacing after macros right
\PassOptionsToPackage{printonlyused,smaller}{acronym}
  \usepackage{acronym} % nice macros for handling all acronyms in the thesis
  %\renewcommand{\bflabel}[1]{{#1}\hfill} % fix the list of acronyms --> no longer working
  %\renewcommand*{\acsfont}[1]{\textsc{#1}}
  %\renewcommand*{\aclabelfont}[1]{\acsfont{#1}}
  %\def\bflabel#1{{#1\hfill}}
  \def\bflabel#1{{\acsfont{#1}\hfill}}
  \def\aclabelfont#1{\acsfont{#1}}
% ****************************************************************************************************
\usepackage{pgfplots} % External TikZ/PGF support (thanks to Andreas Nautsch)
% \usetikzlibrary{external}
% \tikzexternalize[mode=list and make, prefix=ext-tikz/]
% ****************************************************************************************************


% ****************************************************************************************************
% 4. Setup floats: tables, (sub)figures, and captions
% ****************************************************************************************************
\usepackage{tabularx} % better tables
  \setlength{\extrarowheight}{3pt} % increase table row height
\newcommand{\tableheadline}[1]{\multicolumn{1}{l}{\spacedlowsmallcaps{#1}}}
\newcommand{\myfloatalign}{\centering} % to be used with each float for alignment

% ****************************************************************************************************

% ****************************************************************************************************
% 5. Setup code listings
% ****************************************************************************************************
% \usepackage{listings}
\usepackage{longtable}

%\lstset{emph={trueIndex,root},emphstyle=\color{BlueViolet}}%\underbar} % for special keywords
% \lstset{language=[LaTeX]Tex,%C++,
%   morekeywords={PassOptionsToPackage,selectlanguage},
%   keywordstyle=\color{RoyalBlue},%\bfseries,
%   basicstyle=\small\ttfamily,
%   %identifierstyle=\color{NavyBlue},
%   commentstyle=\color{Green}\ttfamily,
%   stringstyle=\rmfamily,
%   numbers=none,%left,%
%   numberstyle=\scriptsize,%\tiny
%   stepnumber=5,
%   numbersep=8pt,
%   showstringspaces=false,
%   breaklines=true,
%   %frameround=ftff,
%   %frame=single,
%   belowcaptionskip=.75\baselineskip
%   %frame=L
% }
% ****************************************************************************************************
\usepackage{setspace}                                                                                               
\usepackage{minted} % Syntax highlighting                                                                                                      
\usemintedstyle{tango}
\definecolor{tango-bg}{HTML}{F8F8F8}

\newminted{python}{bgcolor=tango-bg,frame=lines,framesep=2mm,samepage=true,fontsize=\footnotesize}


% ****************************************************************************************************
% 6. Last calls before the bar closes
% ****************************************************************************************************
% ********************************************************************
% Her Majesty herself
% ********************************************************************
\usepackage{classicthesis}


% ********************************************************************
% Fine-tune hyperreferences (hyperref should be called last)
% ********************************************************************
\hypersetup{%
  %draft, % hyperref's draft mode, for printing see below
  colorlinks=true, linktocpage=true, pdfstartpage=3, pdfstartview=FitV,%
  % uncomment the following line if you want to have black links (e.g., for printing)
  %colorlinks=false, linktocpage=false, pdfstartpage=3, pdfstartview=FitV, pdfborder={0 0 0},%
  breaklinks=true, pageanchor=true,%
  pdfpagemode=UseNone, %
  % pdfpagemode=UseOutlines,%
  plainpages=false, bookmarksnumbered, bookmarksopen=true, bookmarksopenlevel=1,%
  hypertexnames=true, pdfhighlight=/O,%nesting=true,%frenchlinks,%
  urlcolor=CTurl, linkcolor=CTlink, citecolor=CTcitation, %pagecolor=RoyalBlue,%
  %urlcolor=Black, linkcolor=Black, citecolor=Black, %pagecolor=Black,%
  pdftitle={\myTitle},%
  pdfauthor={\textcopyright\ \myName, \myUni, \myFaculty},%
  pdfsubject={},%
  pdfkeywords={},%
  pdfcreator={pdfLaTeX},%
  pdfproducer={LaTeX with hyperref and classicthesis}%
}


% ********************************************************************
% Setup autoreferences (hyperref and babel)
% ********************************************************************
% There are some issues regarding autorefnames
% http://www.tex.ac.uk/cgi-bin/texfaq2html?label=latexwords
% you have to redefine the macros for the
% language you use, e.g., american, ngerman
% (as chosen when loading babel/AtBeginDocument)
% ********************************************************************
\makeatletter
\@ifpackageloaded{babel}%
  {%
    \addto\extrasamerican{%
      \renewcommand*{\figureautorefname}{Figure}%
      \renewcommand*{\tableautorefname}{Table}%
      \renewcommand*{\partautorefname}{Part}%
      \renewcommand*{\chapterautorefname}{Chapter}%
      \renewcommand*{\sectionautorefname}{Section}%
      \renewcommand*{\subsectionautorefname}{Section}%
      \renewcommand*{\subsubsectionautorefname}{Section}%
    }%
    \addto\extrasngerman{%
      \renewcommand*{\paragraphautorefname}{Absatz}%
      \renewcommand*{\subparagraphautorefname}{Unterabsatz}%
      \renewcommand*{\footnoteautorefname}{Fu\"snote}%
      \renewcommand*{\FancyVerbLineautorefname}{Zeile}%
      \renewcommand*{\theoremautorefname}{Theorem}%
      \renewcommand*{\appendixautorefname}{Anhang}%
      \renewcommand*{\equationautorefname}{Gleichung}%
      \renewcommand*{\itemautorefname}{Punkt}%
    }%
      % Fix to getting autorefs for subfigures right (thanks to Belinda Vogt for changing the definition)
      \providecommand{\subfigureautorefname}{\figureautorefname}%
    }{\relax}
\makeatother


% ********************************************************************
% Development Stuff
% ********************************************************************
\listfiles
%\PassOptionsToPackage{l2tabu,orthodox,abort}{nag}
%  \usepackage{nag}
%\PassOptionsToPackage{warning, all}{onlyamsmath}
%  \usepackage{onlyamsmath}


% ****************************************************************************************************
% 7. Further adjustments (experimental)
% ****************************************************************************************************
% ********************************************************************
% Changing the text area
% ********************************************************************
%\areaset[current]{312pt}{761pt} % 686 (factor 2.2) + 33 head + 42 head \the\footskip
%\setlength{\marginparwidth}{7em}%
%\setlength{\marginparsep}{2em}%

% ********************************************************************
% Using different fonts
% ********************************************************************
%\usepackage[oldstylenums]{kpfonts} % oldstyle notextcomp
% \usepackage[osf]{libertine}
%\usepackage[light,condensed,math]{iwona}
%\renewcommand{\sfdefault}{iwona}
%\usepackage{lmodern} % <-- no osf support :-(
%\usepackage{cfr-lm} %
%\usepackage[urw-garamond]{mathdesign} <-- no osf support :-(
%\usepackage[default,osfigures]{opensans} % scale=0.95
% \usepackage[sfdefault]{FiraSans}
% \usepackage[opticals,mathlf]{MinionPro} % onlytext
% ********************************************************************
%\usepackage[largesc,osf]{newpxtext}
%\linespread{1.05} % a bit more for Palatino
% Used to fix these:
% https://bitbucket.org/amiede/classicthesis/issues/139/italics-in-pallatino-capitals-chapter
% https://bitbucket.org/amiede/classicthesis/issues/45/problema-testatine-su-classicthesis-style
% ********************************************************************
% ****************************************************************************************************

\newcommand{\contribs}[1]{
  \begin{minipage}{.3\linewidth}
  \end{minipage}
  \hfill
  \begin{minipage}{.67\linewidth}
    %\begin{flushright}
      \begin{small}
        \textcolor{gray}{\textsf{#1}}
      \end{small}
      \vspace{20pt}\\
    %\end{flushright}
  \end{minipage}
}

\input{mathcommands.tex}

\usepackage{siunitx}

\floatstyle{ruled}
\newfloat{algorithm}{tbp}{loa}
\providecommand{\algorithmname}{Algorithm}
\floatname{algorithm}{\protect\algorithmname}

% Example definitions.
\def\x{{\mathbf{x}}}
\def\L{{\cal{L}}}
\def\Pitch{{F}}

% Official addiontional AudioLabs colors
\definecolor{nice}{rgb}{0.8,0.725490196078431,0.454901960784314}
\definecolor{almagenta}{RGB}{210,85,255}
\definecolor{alturquise}{RGB}{43,174,91}
\definecolor{algreen}{RGB}{111,217,0}
\definecolor{alturquisedark}{RGB}{0,151,164}

% Example definitions.
% --------------------
\providecommand{\e}[1]{\ensuremath{\times 10^{#1}}}


% \renewcommand\formatchapter[1]{%
%     \setbox0=\hbox{\chapterNumber\thechapter\hspace{10pt}\vline\ }%
%     \begin{minipage}[t]{\dimexpr\linewidth-\wd0\relax}%
%     \raggedright\spacedallcaps{#1}%
%     \end{minipage}%
% }

% ****************************************************************************************************
% If you like the classicthesis, then I would appreciate a postcard.
% My address can be found in the file ClassicThesis.pdf. A collection
% of the postcards I received so far is available online at
% http://postcards.miede.de
% ****************************************************************************************************


% ****************************************************************************************************
% 0. Set the encoding of your files. UTF-8 is the only sensible encoding nowadays. If you can't read
% äöüßáéçèê∂åëæƒÏ€ then change the encoding setting in your editor, not the line below. If your editor
% does not support utf8 use another editor!
% ****************************************************************************************************
\PassOptionsToPackage{utf8}{inputenc}
  \usepackage{inputenc}

\PassOptionsToPackage{T1}{fontenc} % T2A for cyrillics
  \usepackage{fontenc}


% ****************************************************************************************************
% 1. Configure classicthesis for your needs here, e.g., remove "drafting" below
% in order to deactivate the time-stamp on the pages
% (see ClassicThesis.pdf for more information):
% ****************************************************************************************************
\PassOptionsToPackage{
  drafting=true,    % print version information on the bottom of the pages
  tocaligned=false, % the left column of the toc will be aligned (no indentation)
  dottedtoc=false,  % page numbers in ToC flushed right
  eulerchapternumbers=true, % use AMS Euler for chapter font (otherwise Palatino)
  linedheaders=false,       % chaper headers will have line above and beneath
  floatperchapter=true,     % numbering per chapter for all floats (i.e., Figure 1.1)
  eulermath=false,  % use awesome Euler fonts for mathematical formulae (only with pdfLaTeX)
  beramono=true,    % toggle a nice monospaced font (w/ bold)
  palatino=true,    % deactivate standard font for loading another one, see the last section at the end of this file for suggestions
  style=arsclassica % classicthesis, arsclassica
}{classicthesis}


% ****************************************************************************************************
% 2. Personal data and user ad-hoc commands (insert your own data here)
% ****************************************************************************************************
% Autogenerated, do not edit
\newcommand{\revisiondate}{2020-01-17}
\newcommand{\revision}{5d7b1d0}


% Separation and Count Estimation for Audio Signals Overlapping in Time and Frequency
% Trennung und Schätzung der Anzahl von Zeit- und Frequenzüberlappenden Audiosignalen

\newcommand{\myTitle}{Separation and Count Estimation for Audio Sources Overlapping in Time and Frequency\xspace}
\newcommand{\myTitleGerman}{Trennung und Schätzung der Anzahl von Audiosignalquellen mit Zeit- und Frequenzüberlappung\xspace}
\newcommand{\mySubtitle}{test\xspace}
\newcommand{\myDegree}{Doktor-Ingenieur (Dr.-Ing.)\xspace}
\newcommand{\myName}{Fabian-Robert Stöter\xspace}
\newcommand{\myProf}{Prof. Dr.-Ing. Bernd Edler\xspace}
\newcommand{\myOtherProf}{Put name here\xspace}
\newcommand{\mySupervisor}{Put name here\xspace}
\newcommand{\myFaculty}{International Audio Laboratories Erlangen\xspace}
\newcommand{\myDepartment}{\xspace}
\newcommand{\myUni}{Universität Erlangen-Nürnberg\xspace}
\newcommand{\myLocation}{Montpellier\xspace}
\newcommand{\myTime}{22. November 2018\xspace}
\newcommand{\myVersion}{commit \revision}

% ********************************************************************
% Setup, finetuning, and useful commands
% ********************************************************************
\providecommand{\mLyX}{L\kern-.1667em\lower.25em\hbox{Y}\kern-.125emX\@}
\newcommand{\ie}{i.\,e.}
\newcommand{\Ie}{I.\,e.}
\newcommand{\eg}{e.\,g.}
\newcommand{\Eg}{E.\,g.}
% ****************************************************************************************************


% ****************************************************************************************************
% 3. Loading some handy packages
% ****************************************************************************************************
% ********************************************************************
% Packages with options that might require adjustments
% ********************************************************************
\PassOptionsToPackage{ngerman,american}{babel} % change this to your language(s), main language last
% Spanish languages need extra options in order to work with this template
%\PassOptionsToPackage{spanish,es-lcroman}{babel}
    \usepackage{babel}

\usepackage{csquotes}
\PassOptionsToPackage{%
  backend=biber,bibencoding=utf8, %instead of bibtex
  % backend=bibtex8,bibencoding=ascii,%
  language=auto,%
  % style=ieee,%
  % dashed=false,
  defernumbers,
  style=alphabetic, % Author 1999, 2010
  %bibstyle=authoryear,dashed=false, % dashed: substitute rep. author with ---
  sorting=nyt, % name, year, title
  maxbibnames=10, % default: 3, et al.
  %backref=true,%
  natbib=false % natbib compatibility mode (\citep and \citet still work)
}{biblatex}

\usepackage{biblatex}

\ExecuteBibliographyOptions{%
  sorting=nty, % Sort by name, title, year.
  maxbibnames=15,%
}%

% Emphasize own name in References with boldface
\usepackage{xpatch}

\usepackage{amsmath}

\usepackage{float}
\usepackage{amsmath,amssymb,amsfonts,amsthm}
\usepackage{euscript}

%----------------------
\usepackage{tikz}
\usetikzlibrary{positioning,fit, calc}
\usepackage{pgfplots}

\usepackage{bm}

% ********************************************************************
% General useful packages
% ********************************************************************
\usepackage{graphicx} %
\usepackage{float}
\usepackage{scrhack} % fix warnings when using KOMA with listings package
\usepackage{xspace} % to get the spacing after macros right
\PassOptionsToPackage{printonlyused,smaller}{acronym}
  \usepackage{acronym} % nice macros for handling all acronyms in the thesis
  %\renewcommand{\bflabel}[1]{{#1}\hfill} % fix the list of acronyms --> no longer working
  %\renewcommand*{\acsfont}[1]{\textsc{#1}}
  %\renewcommand*{\aclabelfont}[1]{\acsfont{#1}}
  %\def\bflabel#1{{#1\hfill}}
  \def\bflabel#1{{\acsfont{#1}\hfill}}
  \def\aclabelfont#1{\acsfont{#1}}
% ****************************************************************************************************
\usepackage{pgfplots} % External TikZ/PGF support (thanks to Andreas Nautsch)
% \usetikzlibrary{external}
% \tikzexternalize[mode=list and make, prefix=ext-tikz/]
% ****************************************************************************************************


% ****************************************************************************************************
% 4. Setup floats: tables, (sub)figures, and captions
% ****************************************************************************************************
\usepackage{tabularx} % better tables
  \setlength{\extrarowheight}{3pt} % increase table row height
\newcommand{\tableheadline}[1]{\multicolumn{1}{l}{\spacedlowsmallcaps{#1}}}
\newcommand{\myfloatalign}{\centering} % to be used with each float for alignment

% ****************************************************************************************************

% ****************************************************************************************************
% 5. Setup code listings
% ****************************************************************************************************
% \usepackage{listings}
\usepackage{longtable}

%\lstset{emph={trueIndex,root},emphstyle=\color{BlueViolet}}%\underbar} % for special keywords
% \lstset{language=[LaTeX]Tex,%C++,
%   morekeywords={PassOptionsToPackage,selectlanguage},
%   keywordstyle=\color{RoyalBlue},%\bfseries,
%   basicstyle=\small\ttfamily,
%   %identifierstyle=\color{NavyBlue},
%   commentstyle=\color{Green}\ttfamily,
%   stringstyle=\rmfamily,
%   numbers=none,%left,%
%   numberstyle=\scriptsize,%\tiny
%   stepnumber=5,
%   numbersep=8pt,
%   showstringspaces=false,
%   breaklines=true,
%   %frameround=ftff,
%   %frame=single,
%   belowcaptionskip=.75\baselineskip
%   %frame=L
% }
% ****************************************************************************************************
\usepackage{setspace}                                                                                               
\usepackage{minted} % Syntax highlighting                                                                                                      
\usemintedstyle{tango}
\definecolor{tango-bg}{HTML}{F8F8F8}

\newminted{python}{bgcolor=tango-bg,frame=lines,framesep=2mm,samepage=true,fontsize=\footnotesize}


% ****************************************************************************************************
% 6. Last calls before the bar closes
% ****************************************************************************************************
% ********************************************************************
% Her Majesty herself
% ********************************************************************
\usepackage{classicthesis}


% ********************************************************************
% Fine-tune hyperreferences (hyperref should be called last)
% ********************************************************************
\hypersetup{%
  %draft, % hyperref's draft mode, for printing see below
  colorlinks=true, linktocpage=true, pdfstartpage=3, pdfstartview=FitV,%
  % uncomment the following line if you want to have black links (e.g., for printing)
  %colorlinks=false, linktocpage=false, pdfstartpage=3, pdfstartview=FitV, pdfborder={0 0 0},%
  breaklinks=true, pageanchor=true,%
  pdfpagemode=UseNone, %
  % pdfpagemode=UseOutlines,%
  plainpages=false, bookmarksnumbered, bookmarksopen=true, bookmarksopenlevel=1,%
  hypertexnames=true, pdfhighlight=/O,%nesting=true,%frenchlinks,%
  urlcolor=CTurl, linkcolor=CTlink, citecolor=CTcitation, %pagecolor=RoyalBlue,%
  %urlcolor=Black, linkcolor=Black, citecolor=Black, %pagecolor=Black,%
  pdftitle={\myTitle},%
  pdfauthor={\textcopyright\ \myName, \myUni, \myFaculty},%
  pdfsubject={},%
  pdfkeywords={},%
  pdfcreator={pdfLaTeX},%
  pdfproducer={LaTeX with hyperref and classicthesis}%
}


% ********************************************************************
% Setup autoreferences (hyperref and babel)
% ********************************************************************
% There are some issues regarding autorefnames
% http://www.tex.ac.uk/cgi-bin/texfaq2html?label=latexwords
% you have to redefine the macros for the
% language you use, e.g., american, ngerman
% (as chosen when loading babel/AtBeginDocument)
% ********************************************************************
\makeatletter
\@ifpackageloaded{babel}%
  {%
    \addto\extrasamerican{%
      \renewcommand*{\figureautorefname}{Figure}%
      \renewcommand*{\tableautorefname}{Table}%
      \renewcommand*{\partautorefname}{Part}%
      \renewcommand*{\chapterautorefname}{Chapter}%
      \renewcommand*{\sectionautorefname}{Section}%
      \renewcommand*{\subsectionautorefname}{Section}%
      \renewcommand*{\subsubsectionautorefname}{Section}%
    }%
    \addto\extrasngerman{%
      \renewcommand*{\paragraphautorefname}{Absatz}%
      \renewcommand*{\subparagraphautorefname}{Unterabsatz}%
      \renewcommand*{\footnoteautorefname}{Fu\"snote}%
      \renewcommand*{\FancyVerbLineautorefname}{Zeile}%
      \renewcommand*{\theoremautorefname}{Theorem}%
      \renewcommand*{\appendixautorefname}{Anhang}%
      \renewcommand*{\equationautorefname}{Gleichung}%
      \renewcommand*{\itemautorefname}{Punkt}%
    }%
      % Fix to getting autorefs for subfigures right (thanks to Belinda Vogt for changing the definition)
      \providecommand{\subfigureautorefname}{\figureautorefname}%
    }{\relax}
\makeatother


% ********************************************************************
% Development Stuff
% ********************************************************************
\listfiles
%\PassOptionsToPackage{l2tabu,orthodox,abort}{nag}
%  \usepackage{nag}
%\PassOptionsToPackage{warning, all}{onlyamsmath}
%  \usepackage{onlyamsmath}


% ****************************************************************************************************
% 7. Further adjustments (experimental)
% ****************************************************************************************************
% ********************************************************************
% Changing the text area
% ********************************************************************
%\areaset[current]{312pt}{761pt} % 686 (factor 2.2) + 33 head + 42 head \the\footskip
%\setlength{\marginparwidth}{7em}%
%\setlength{\marginparsep}{2em}%

% ********************************************************************
% Using different fonts
% ********************************************************************
%\usepackage[oldstylenums]{kpfonts} % oldstyle notextcomp
% \usepackage[osf]{libertine}
%\usepackage[light,condensed,math]{iwona}
%\renewcommand{\sfdefault}{iwona}
%\usepackage{lmodern} % <-- no osf support :-(
%\usepackage{cfr-lm} %
%\usepackage[urw-garamond]{mathdesign} <-- no osf support :-(
%\usepackage[default,osfigures]{opensans} % scale=0.95
% \usepackage[sfdefault]{FiraSans}
% \usepackage[opticals,mathlf]{MinionPro} % onlytext
% ********************************************************************
%\usepackage[largesc,osf]{newpxtext}
%\linespread{1.05} % a bit more for Palatino
% Used to fix these:
% https://bitbucket.org/amiede/classicthesis/issues/139/italics-in-pallatino-capitals-chapter
% https://bitbucket.org/amiede/classicthesis/issues/45/problema-testatine-su-classicthesis-style
% ********************************************************************
% ****************************************************************************************************

\newcommand{\contribs}[1]{
  \begin{minipage}{.3\linewidth}
  \end{minipage}
  \hfill
  \begin{minipage}{.67\linewidth}
    %\begin{flushright}
      \begin{small}
        \textcolor{gray}{\textsf{#1}}
      \end{small}
      \vspace{20pt}\\
    %\end{flushright}
  \end{minipage}
}

\input{mathcommands.tex}

\usepackage{siunitx}

\floatstyle{ruled}
\newfloat{algorithm}{tbp}{loa}
\providecommand{\algorithmname}{Algorithm}
\floatname{algorithm}{\protect\algorithmname}

% Example definitions.
\def\x{{\mathbf{x}}}
\def\L{{\cal{L}}}
\def\Pitch{{F}}

% Official addiontional AudioLabs colors
\definecolor{nice}{rgb}{0.8,0.725490196078431,0.454901960784314}
\definecolor{almagenta}{RGB}{210,85,255}
\definecolor{alturquise}{RGB}{43,174,91}
\definecolor{algreen}{RGB}{111,217,0}
\definecolor{alturquisedark}{RGB}{0,151,164}

% Example definitions.
% --------------------
\providecommand{\e}[1]{\ensuremath{\times 10^{#1}}}


% \renewcommand\formatchapter[1]{%
%     \setbox0=\hbox{\chapterNumber\thechapter\hspace{10pt}\vline\ }%
%     \begin{minipage}[t]{\dimexpr\linewidth-\wd0\relax}%
%     \raggedright\spacedallcaps{#1}%
%     \end{minipage}%
% }

% ****************************************************************************************************
% If you like the classicthesis, then I would appreciate a postcard.
% My address can be found in the file ClassicThesis.pdf. A collection
% of the postcards I received so far is available online at
% http://postcards.miede.de
% ****************************************************************************************************


% ****************************************************************************************************
% 0. Set the encoding of your files. UTF-8 is the only sensible encoding nowadays. If you can't read
% äöüßáéçèê∂åëæƒÏ€ then change the encoding setting in your editor, not the line below. If your editor
% does not support utf8 use another editor!
% ****************************************************************************************************
\PassOptionsToPackage{utf8}{inputenc}
  \usepackage{inputenc}

\PassOptionsToPackage{T1}{fontenc} % T2A for cyrillics
  \usepackage{fontenc}


% ****************************************************************************************************
% 1. Configure classicthesis for your needs here, e.g., remove "drafting" below
% in order to deactivate the time-stamp on the pages
% (see ClassicThesis.pdf for more information):
% ****************************************************************************************************
\PassOptionsToPackage{
  drafting=true,    % print version information on the bottom of the pages
  tocaligned=false, % the left column of the toc will be aligned (no indentation)
  dottedtoc=false,  % page numbers in ToC flushed right
  eulerchapternumbers=true, % use AMS Euler for chapter font (otherwise Palatino)
  linedheaders=false,       % chaper headers will have line above and beneath
  floatperchapter=true,     % numbering per chapter for all floats (i.e., Figure 1.1)
  eulermath=false,  % use awesome Euler fonts for mathematical formulae (only with pdfLaTeX)
  beramono=true,    % toggle a nice monospaced font (w/ bold)
  palatino=true,    % deactivate standard font for loading another one, see the last section at the end of this file for suggestions
  style=arsclassica % classicthesis, arsclassica
}{classicthesis}


% ****************************************************************************************************
% 2. Personal data and user ad-hoc commands (insert your own data here)
% ****************************************************************************************************
% Autogenerated, do not edit
\newcommand{\revisiondate}{2020-01-17}
\newcommand{\revision}{5d7b1d0}


% Separation and Count Estimation for Audio Signals Overlapping in Time and Frequency
% Trennung und Schätzung der Anzahl von Zeit- und Frequenzüberlappenden Audiosignalen

\newcommand{\myTitle}{Separation and Count Estimation for Audio Sources Overlapping in Time and Frequency\xspace}
\newcommand{\myTitleGerman}{Trennung und Schätzung der Anzahl von Audiosignalquellen mit Zeit- und Frequenzüberlappung\xspace}
\newcommand{\mySubtitle}{test\xspace}
\newcommand{\myDegree}{Doktor-Ingenieur (Dr.-Ing.)\xspace}
\newcommand{\myName}{Fabian-Robert Stöter\xspace}
\newcommand{\myProf}{Prof. Dr.-Ing. Bernd Edler\xspace}
\newcommand{\myOtherProf}{Put name here\xspace}
\newcommand{\mySupervisor}{Put name here\xspace}
\newcommand{\myFaculty}{International Audio Laboratories Erlangen\xspace}
\newcommand{\myDepartment}{\xspace}
\newcommand{\myUni}{Universität Erlangen-Nürnberg\xspace}
\newcommand{\myLocation}{Montpellier\xspace}
\newcommand{\myTime}{22. November 2018\xspace}
\newcommand{\myVersion}{commit \revision}

% ********************************************************************
% Setup, finetuning, and useful commands
% ********************************************************************
\providecommand{\mLyX}{L\kern-.1667em\lower.25em\hbox{Y}\kern-.125emX\@}
\newcommand{\ie}{i.\,e.}
\newcommand{\Ie}{I.\,e.}
\newcommand{\eg}{e.\,g.}
\newcommand{\Eg}{E.\,g.}
% ****************************************************************************************************


% ****************************************************************************************************
% 3. Loading some handy packages
% ****************************************************************************************************
% ********************************************************************
% Packages with options that might require adjustments
% ********************************************************************
\PassOptionsToPackage{ngerman,american}{babel} % change this to your language(s), main language last
% Spanish languages need extra options in order to work with this template
%\PassOptionsToPackage{spanish,es-lcroman}{babel}
    \usepackage{babel}

\usepackage{csquotes}
\PassOptionsToPackage{%
  backend=biber,bibencoding=utf8, %instead of bibtex
  % backend=bibtex8,bibencoding=ascii,%
  language=auto,%
  % style=ieee,%
  % dashed=false,
  defernumbers,
  style=alphabetic, % Author 1999, 2010
  %bibstyle=authoryear,dashed=false, % dashed: substitute rep. author with ---
  sorting=nyt, % name, year, title
  maxbibnames=10, % default: 3, et al.
  %backref=true,%
  natbib=false % natbib compatibility mode (\citep and \citet still work)
}{biblatex}

\usepackage{biblatex}

\ExecuteBibliographyOptions{%
  sorting=nty, % Sort by name, title, year.
  maxbibnames=15,%
}%

% Emphasize own name in References with boldface
\usepackage{xpatch}

\usepackage{amsmath}

\usepackage{float}
\usepackage{amsmath,amssymb,amsfonts,amsthm}
\usepackage{euscript}

%----------------------
\usepackage{tikz}
\usetikzlibrary{positioning,fit, calc}
\usepackage{pgfplots}

\usepackage{bm}

% ********************************************************************
% General useful packages
% ********************************************************************
\usepackage{graphicx} %
\usepackage{float}
\usepackage{scrhack} % fix warnings when using KOMA with listings package
\usepackage{xspace} % to get the spacing after macros right
\PassOptionsToPackage{printonlyused,smaller}{acronym}
  \usepackage{acronym} % nice macros for handling all acronyms in the thesis
  %\renewcommand{\bflabel}[1]{{#1}\hfill} % fix the list of acronyms --> no longer working
  %\renewcommand*{\acsfont}[1]{\textsc{#1}}
  %\renewcommand*{\aclabelfont}[1]{\acsfont{#1}}
  %\def\bflabel#1{{#1\hfill}}
  \def\bflabel#1{{\acsfont{#1}\hfill}}
  \def\aclabelfont#1{\acsfont{#1}}
% ****************************************************************************************************
\usepackage{pgfplots} % External TikZ/PGF support (thanks to Andreas Nautsch)
% \usetikzlibrary{external}
% \tikzexternalize[mode=list and make, prefix=ext-tikz/]
% ****************************************************************************************************


% ****************************************************************************************************
% 4. Setup floats: tables, (sub)figures, and captions
% ****************************************************************************************************
\usepackage{tabularx} % better tables
  \setlength{\extrarowheight}{3pt} % increase table row height
\newcommand{\tableheadline}[1]{\multicolumn{1}{l}{\spacedlowsmallcaps{#1}}}
\newcommand{\myfloatalign}{\centering} % to be used with each float for alignment

% ****************************************************************************************************

% ****************************************************************************************************
% 5. Setup code listings
% ****************************************************************************************************
% \usepackage{listings}
\usepackage{longtable}

%\lstset{emph={trueIndex,root},emphstyle=\color{BlueViolet}}%\underbar} % for special keywords
% \lstset{language=[LaTeX]Tex,%C++,
%   morekeywords={PassOptionsToPackage,selectlanguage},
%   keywordstyle=\color{RoyalBlue},%\bfseries,
%   basicstyle=\small\ttfamily,
%   %identifierstyle=\color{NavyBlue},
%   commentstyle=\color{Green}\ttfamily,
%   stringstyle=\rmfamily,
%   numbers=none,%left,%
%   numberstyle=\scriptsize,%\tiny
%   stepnumber=5,
%   numbersep=8pt,
%   showstringspaces=false,
%   breaklines=true,
%   %frameround=ftff,
%   %frame=single,
%   belowcaptionskip=.75\baselineskip
%   %frame=L
% }
% ****************************************************************************************************
\usepackage{setspace}                                                                                               
\usepackage{minted} % Syntax highlighting                                                                                                      
\usemintedstyle{tango}
\definecolor{tango-bg}{HTML}{F8F8F8}

\newminted{python}{bgcolor=tango-bg,frame=lines,framesep=2mm,samepage=true,fontsize=\footnotesize}


% ****************************************************************************************************
% 6. Last calls before the bar closes
% ****************************************************************************************************
% ********************************************************************
% Her Majesty herself
% ********************************************************************
\usepackage{classicthesis}


% ********************************************************************
% Fine-tune hyperreferences (hyperref should be called last)
% ********************************************************************
\hypersetup{%
  %draft, % hyperref's draft mode, for printing see below
  colorlinks=true, linktocpage=true, pdfstartpage=3, pdfstartview=FitV,%
  % uncomment the following line if you want to have black links (e.g., for printing)
  %colorlinks=false, linktocpage=false, pdfstartpage=3, pdfstartview=FitV, pdfborder={0 0 0},%
  breaklinks=true, pageanchor=true,%
  pdfpagemode=UseNone, %
  % pdfpagemode=UseOutlines,%
  plainpages=false, bookmarksnumbered, bookmarksopen=true, bookmarksopenlevel=1,%
  hypertexnames=true, pdfhighlight=/O,%nesting=true,%frenchlinks,%
  urlcolor=CTurl, linkcolor=CTlink, citecolor=CTcitation, %pagecolor=RoyalBlue,%
  %urlcolor=Black, linkcolor=Black, citecolor=Black, %pagecolor=Black,%
  pdftitle={\myTitle},%
  pdfauthor={\textcopyright\ \myName, \myUni, \myFaculty},%
  pdfsubject={},%
  pdfkeywords={},%
  pdfcreator={pdfLaTeX},%
  pdfproducer={LaTeX with hyperref and classicthesis}%
}


% ********************************************************************
% Setup autoreferences (hyperref and babel)
% ********************************************************************
% There are some issues regarding autorefnames
% http://www.tex.ac.uk/cgi-bin/texfaq2html?label=latexwords
% you have to redefine the macros for the
% language you use, e.g., american, ngerman
% (as chosen when loading babel/AtBeginDocument)
% ********************************************************************
\makeatletter
\@ifpackageloaded{babel}%
  {%
    \addto\extrasamerican{%
      \renewcommand*{\figureautorefname}{Figure}%
      \renewcommand*{\tableautorefname}{Table}%
      \renewcommand*{\partautorefname}{Part}%
      \renewcommand*{\chapterautorefname}{Chapter}%
      \renewcommand*{\sectionautorefname}{Section}%
      \renewcommand*{\subsectionautorefname}{Section}%
      \renewcommand*{\subsubsectionautorefname}{Section}%
    }%
    \addto\extrasngerman{%
      \renewcommand*{\paragraphautorefname}{Absatz}%
      \renewcommand*{\subparagraphautorefname}{Unterabsatz}%
      \renewcommand*{\footnoteautorefname}{Fu\"snote}%
      \renewcommand*{\FancyVerbLineautorefname}{Zeile}%
      \renewcommand*{\theoremautorefname}{Theorem}%
      \renewcommand*{\appendixautorefname}{Anhang}%
      \renewcommand*{\equationautorefname}{Gleichung}%
      \renewcommand*{\itemautorefname}{Punkt}%
    }%
      % Fix to getting autorefs for subfigures right (thanks to Belinda Vogt for changing the definition)
      \providecommand{\subfigureautorefname}{\figureautorefname}%
    }{\relax}
\makeatother


% ********************************************************************
% Development Stuff
% ********************************************************************
\listfiles
%\PassOptionsToPackage{l2tabu,orthodox,abort}{nag}
%  \usepackage{nag}
%\PassOptionsToPackage{warning, all}{onlyamsmath}
%  \usepackage{onlyamsmath}


% ****************************************************************************************************
% 7. Further adjustments (experimental)
% ****************************************************************************************************
% ********************************************************************
% Changing the text area
% ********************************************************************
%\areaset[current]{312pt}{761pt} % 686 (factor 2.2) + 33 head + 42 head \the\footskip
%\setlength{\marginparwidth}{7em}%
%\setlength{\marginparsep}{2em}%

% ********************************************************************
% Using different fonts
% ********************************************************************
%\usepackage[oldstylenums]{kpfonts} % oldstyle notextcomp
% \usepackage[osf]{libertine}
%\usepackage[light,condensed,math]{iwona}
%\renewcommand{\sfdefault}{iwona}
%\usepackage{lmodern} % <-- no osf support :-(
%\usepackage{cfr-lm} %
%\usepackage[urw-garamond]{mathdesign} <-- no osf support :-(
%\usepackage[default,osfigures]{opensans} % scale=0.95
% \usepackage[sfdefault]{FiraSans}
% \usepackage[opticals,mathlf]{MinionPro} % onlytext
% ********************************************************************
%\usepackage[largesc,osf]{newpxtext}
%\linespread{1.05} % a bit more for Palatino
% Used to fix these:
% https://bitbucket.org/amiede/classicthesis/issues/139/italics-in-pallatino-capitals-chapter
% https://bitbucket.org/amiede/classicthesis/issues/45/problema-testatine-su-classicthesis-style
% ********************************************************************
% ****************************************************************************************************

\newcommand{\contribs}[1]{
  \begin{minipage}{.3\linewidth}
  \end{minipage}
  \hfill
  \begin{minipage}{.67\linewidth}
    %\begin{flushright}
      \begin{small}
        \textcolor{gray}{\textsf{#1}}
      \end{small}
      \vspace{20pt}\\
    %\end{flushright}
  \end{minipage}
}

\input{mathcommands.tex}

\usepackage{siunitx}

\floatstyle{ruled}
\newfloat{algorithm}{tbp}{loa}
\providecommand{\algorithmname}{Algorithm}
\floatname{algorithm}{\protect\algorithmname}

% Example definitions.
\def\x{{\mathbf{x}}}
\def\L{{\cal{L}}}
\def\Pitch{{F}}

% Official addiontional AudioLabs colors
\definecolor{nice}{rgb}{0.8,0.725490196078431,0.454901960784314}
\definecolor{almagenta}{RGB}{210,85,255}
\definecolor{alturquise}{RGB}{43,174,91}
\definecolor{algreen}{RGB}{111,217,0}
\definecolor{alturquisedark}{RGB}{0,151,164}

% Example definitions.
% --------------------
\providecommand{\e}[1]{\ensuremath{\times 10^{#1}}}


% \renewcommand\formatchapter[1]{%
%     \setbox0=\hbox{\chapterNumber\thechapter\hspace{10pt}\vline\ }%
%     \begin{minipage}[t]{\dimexpr\linewidth-\wd0\relax}%
%     \raggedright\spacedallcaps{#1}%
%     \end{minipage}%
% }
