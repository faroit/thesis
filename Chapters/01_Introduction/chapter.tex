%************************************************
\chapter{Introduction}\label{ch:introduction}
%************************************************
%AN EXAMPLE
It is very likely that you know the following situation: you were at a crowded party and the next day your best friend, who was unable to join, asked you ``How many people were there?''.
You then struggled to find an answer because you had such intense conversations with the guests that you were unable to put your focus on the other guests.
It turns out that this scene includes many interesting aspects that are relevant to this thesis. It notably reminds us that in our daily life, we are exposed to situations where multiple events overlap in time.

%INTRODUCING separation
\par
In the proposed example of the conversation at a party, it is evident that our field of view is limited and we can not see the other guests in a crowd.
In contrast to our vision, we physically might have been able to \emph{hear} all sounds, but we deliberatively chose to \emph{listen to} (focus on) only a few sources and attenuate others.
Humans are known to be very good at performing such a task.
In such a noisy environment, we can steer our attention to one sound source, even without eye contact and using only a single ear~\cite{bregman90}.
However, this attention mechanism prevents us from observing the acoustic scene as a whole.

This ability to concentrate on a single source is not limited to just conversations. It also apply to music.
If we imagine attending a music concert of our favorite band, it is likely that we focus on the lead vocalist and miss out many details of the background band, demonstrating again our ability to \emph{separate} audio mixtures, at least cognitively.

\par
In the audio research community, building a machine that is capable of attenuating undesired speakers, when multiple concurrent speakers are present, is known as the ``cocktail party problem''~\cite{haykin05}.
Since almost 70 years (~\cite{cherry53}), research is fascinated by this idea to computationally immitate this ability of humans to separate the sources in a mixture.
In the general setting, which is not restricted to the cocktail party scenario but also includes music processing, this problem is called \emph{source separation} and is one of the key topics considered in this thesis.

%INTRODUCING COUNTING
Our example also highlights that before the actual content for the separate conversations at the party, the \emph{number of sources} may already be a valuable information to describe an audio scene and it is the second core topic of investigation considered in this thesis.

As we exemplify in our work, humans are not impressively good for counting sources based on audio. The fact is that they can only robustly count the number of speakers in a crown up to three, above which they'd rather talk of ```many```.
This sharply contrasts with their mastery for focusing on a single source in a crowded audio scene.
\par
That finding that humans are not experts in source counting suggests that it may be challenging in its own right to build a machine that solves this problem, although that topic appears much less investigated than separation. %YOU SHOULD PROBABLY MENTION SOME REFERENCE HERE ALL THE SAME
Furthermore, from a more technical point of view, many separation methods rely on the prior knowledge of the number of sources for their computations, requiring this information to be estimated beforehand, or provided by a user.
% you may cite our work with gael and durrieu at wiamis 2013 here if you'd like to polish gael (overview of informed source separation)


% NOW WE SWITCH TO THE CONCEPT OF OVERLAP
\par
Our starting point in this thesis is that regardless we consider music or speech mixtures, the core concept that makes processing challenging is the \emph{overlap} of the signals. Our whole work then revolved around addressing separation and counting in cases of extreme signals overlap.
% why?
For this reason, our running scenario will consist in cases where sources almost entirely overlap both in time and in frequency: crowds of people talking all together, or multiple musical instruments playing the same note (in unison).

In this setting of strong overlaps, observable differences between sources are difficult to obtain at a short time scale. However, the key fact we exploit throughout this work is that differences do appear when we consider longer time contexts, for which the variations of sources over time do make a difference.
For instance, speech and the sound of musical instruments can have distinct modulations such as vibrato, created by different physical resonators. In speech, vibratos may actually be achieved on purpose, to make a sound more pleasant to the listener~\cite{fletcher01}.
\par
In this thesis, we show how this concept of ```modulations``` can be used to separate highly overlapped speech and music source signals and
to successfuly achieve separation and counting in such cases.
For this purpose, we first study and develop new representations that improve the analysis and processing of such signals.
Second, we develop new methods built upon such representations to address source separation. Third, we show how features that may be unedrstood as modulation-based play a fundamental role in data-driven models that achieve source counting.
Finally, we show how such research may be transferred from synthetic signals to real-world scenarios 

% TODO: check if scope is clear (not promise too much)

% [X] Tell a story, and tell it well
% [X] Tell the reader the problem you are tackling in this project.
% [ ] Quote data sources, e.g., industry analysts, market surveys, case studies
% [ ] Use plenty of concrete examples (or a running example) and figures
% [X] State clearly how you aim to deal with this problem.
% [ ] Limit the scope of your study.

\section{Summary of Contributions}

This thesis comes with five main contributions:

\begin{enumerate}
\item I review scenarios of time and frequency overlapped audio sources.
I designed a novel scenario where instruments are highly overlapped (unison) but I also consider known scenarios for speech and music.
In this unison scenario, I review, how slowly varying tempo-spectral modulations, caused e.g. by vibrato, can be utilized for separation and source count estimation of highly overlapped signals.
Furthermore I show how these scenarios can stimulate new research directions to \emph{analyze} and \emph{process} such signals.

\item I designed two novel methods to \emph{separate} unison instrument mixtures: one is informed by an estimate of the fundamental frequency variation.
The other is unsupervised, inspired by the way how humans segregate time-varying sources.
Along the way, I also proposed an post-processing to improve \(F0\) estimates based on the same principles.
Next, I study how the observations from the unison scenario can be transferred to real world scenarios such as lead accompaniment separation by applying the deep learning framework.

\item I conducted two detailed experimental studies to assess how humans perceive highly overlapped mixtures and how they perform when asked to estimate the number of sources.
In these studies I focussed on scenarios of overlapped speech as well as polyphonic music recordings.
Both studies confirmed earlier studies, indicating that humans can only correctly estimate the number of concurrent sources up to three.

\item We designed a method to automatically estimate the maximum number of concurrent speakers. This method uses deep neural networks to addresses ``cocktail party'' like environments.
Furthermore, I show that this model reached state-of-the art performance when compared to other models and also supersedes human performance when compared with the results of my subjective listening experiments.
Finally I revealed the relation between slow modulations in speech and the ability of a model of \emph{learning to count}.

\item As a practical contribution, I developed tools to assess the quality of separation system by means of interactive web applications.
I helped to create publicly available datasets for separation and \(F0\) estimation.
Furthermore, I  co-organized the international source separation evaluation campaign (SiSEC) to improve sustainability and reproducibility for the research community.
\end{enumerate}

This thesis is based on the previously listed publications. As such, they are cited repeatedly throughout this thesis. For readability reasons, if a section is mainly based on one of these publication, a remark is added at side of the page, instead of citing the same publication exhaustively. Furthermore, it should be noted ––– if not stated otherwise ---  that these contributions in the publications are based on the ideas and research of the author of this thesis.

\section{Structure of this Thesis}

The thesis and its relevant linked publications are organized into six main chapters.
\begin{description}
  \item[Chapter~\ref{cha:fundamentals}] explains the fundamental concepts of audio signals (Section~\ref{sec:specifics-of-audio-signals}) as well as sources and overlapped sounds (Section~\ref{sources-and-mixtures}), relevant for the remainder of this thesis.
  This includes commonly used transformations and signal representations for the task of source separation and source count estimation.
  Furthermore, the process of mixing sound sources as well as its inverse task --- sound source separation --- are explained.
  The chapter also covers basics of fundamental frequency and its variations (Section~\ref{sub:time-variant-audio-signals}) as an important feature for harmonic audio signals.
  Part of this chapter is based on~\cite{rafii18}.
  \item[Chapter~\ref{cha:highly-overlapped-signals}] introduces relevant tasks and applications in the context of highly overlapped sounds.
  Furthermore, the relevance of highly overlapped source scenarios are discussed and the research track of slow  modulations (Section~\ref{exploiting-slow-modulations}) to address separation tasks is proposed.
  \item[Chapter~\ref{cha:datasets}] presents and discusses the importance of data for analysis and evaluation.
  In this chapter synthetic (Section~\ref{}) and realistic (Section~\ref{}) datasets are presented, created during the course of this thesis~\cite{oss_wice, oss_unison, oss_libricount, stoeter15acm, liutkus17}.
  \item[Chapters~\ref{cha:known} and \ref{cha:unknown}] present separation methods that are developed over the course of this thesis.
  This covers techniques that utilize modulation information when available(Chapter 5, ~\cite{stoeter14, stoeter15icassp}) as well as blind methods (Chapter 6~\cite{stoeter16, liutkus17}).
  \item[Chapters~\ref{cha:countanalysis} and ~\ref{cha:countanalysis}] cover the analysis of overlapped sounds. Specifically, it deals with identifying and estimating the number of sources on music~\cite{schoeffler13, stoeter13} and the cocktail party scenario~\cite{stoeter19, stoeter18}.
  \item[Chapter~\ref{cha:conclusion}] concludes this thesis and gives an outlook into future research directions.
\end{description}
