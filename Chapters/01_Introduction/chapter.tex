%************************************************
\chapter{Introduction}\label{ch:introduction}
%************************************************

Lorem ipsum dolor sit amet, consectetur adipisicing elit, sed do eiusmod tempor incididunt ut labore et dolore magna aliqua. Ut enim ad minim veniam, quis nostrud exercitation ullamco laboris nisi ut aliquip ex ea commodo consequat. Duis aute irure dolor in reprehenderit in voluptate velit esse cillum dolore eu fugiat nulla pariatur. Excepteur sint occaecat cupidatat non proident, sunt in culpa qui officia deserunt mollit anim id est laborum.


\section{Motivation}
Lorem ipsum dolor sit amet, consectetur adipisicing elit, sed do eiusmod tempor incididunt ut labore et dolore magna aliqua. Ut enim ad minim veniam, quis nostrud exercitation ullamco laboris nisi ut aliquip ex ea commodo consequat. Duis aute irure dolor in reprehenderit in voluptate velit esse cillum dolore eu fugiat nulla pariatur. Excepteur sint occaecat cupidatat non proident, sunt in culpa qui officia deserunt mollit anim id est laborum.


\clearpage
\section{Summary of Contributions}

This thesis comes with five main contributions:

\begin{enumerate}
\item I discuss scenarios of time and frequency overlapped audio sources.
I consider known scenarios for speech and music but also present an novel scenario where instruments are playing in unison.
In this scenario, I show, how slowly varying tempo-spectral modulations, caused e.g. by vibrato, can be utilized for separation and analysis of unison signal.
Furthermore I show how these scenarios can stimulate new research directions to \emph{analyse} and \emph{process} such sounds.\\

\item I developed two novel methods to \emph{separate} unison instrument mixtures: one is informed by an estimate of the fundamental frequency variation.
The other is unsupervised, inspired by the way how humans segregate time-varying sources.
Finally, I study how the observations from the unison scenario can be transfered to real world scenarios such as the separation of professional produced music.\\

\item I provide two detailed experimental studies to assess how humans perceive highly overlapped mixtures.
In these studies I focussed on the \emph{number of concurrent sources} in scenarios such as a overlapped speech as well as polyphonic music recordings.
In this vein, I results of auditory experiments that studiy the humans ability to detect the maximum number of sources.\\

\item For the task of \emph{estimating the maximum number of concurrent} speakers I developed a state-of-the art method based on deep neural networks that addresses cocktail party like environments.
Furthermore, I show that this model can supersede human performance when compared with the results of my subjective listening experiments.\\

\item Last, but not least, I provided service to the research community by co-organising the international source separation evaluation campaign (SiSEC) and helped out to improve sustainability and reproducibility.
In that vein, I provide open software and data to assess the quality of separation system using web technologies, which is also used throughout the work presented in this thesis.
\end{enumerate}


\section{Structure of this Thesis}

The thesis and its relevant linked publications are organized into 6 main chapters.
\begin{description}
  \item[Chapter~1], this chapter.
  \item[Chapter~2] explains the fundamental concepts of audio signals, sources and overlapped sounds, relevant for the remainder of this thesis. This includes commonly used transformations and signal representations for the task of source separation and analysis of mixtures.
  Furthermore, the process of mixing sound sources as well as its inverse task --- sound source separation --- are explained.
  The chapter also covers basics of fundamental frequency and its variations as an important feature for harmonic audio signals.
  Part of this chapter is based on~\cite{rafii18}.
  \item[Chapter~3] introduces relevant tasks and applications in the context of highly overlapped sounds.
  Furthermore, the relevance of highly overlapped source scenarios are discussed and a new research track of long term slow modulations is proposed.
  \item[Chapter~4] presents and discusses the importance of data for analysis and evaluation.
  In this chapter datasets that were created during the course of this thesis are discussed~\cite{oss_wice, oss_unison, oss_libricount, liutkus17}.
  \item[Chapter~5 and 6] presents separation methods that are developed over the course of this thesis. It covers techniques that uitilize modulation information when available~\cite{stoeter14, stoeter15acm, stoeter15icassp} as well as blind methods~\cite{stoeter16, liutkus17}.
  \item[Chapter~7 and 8] covers the analysis of overlapped sounds. Specifically, it deals with identifying and estimating the number of sources on music~\cite{schoeffler13, stoeter13} and the cocktail party scenario~\cite{stoeter19, stoeter18}.
  \item[Chapter~9] Concludes this thesis and gives a and outlook into future research directions.
\end{description}
