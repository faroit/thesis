\chapter{Conclusion}


focus on the overlap, not on the non-overlapping part.


Contributed two 4 different research fields
- Fundamental Frequency Estimation
- Modulation based Source Separation based on unsupervised
- Supervised separation
- Deep Learning on Audio

- we now have gathered enough data. Can we establish source count estimation as a separation metric which is way easier to assess compared to MUSHRA tests.

 What did I achieve?
I showed that highly overlapped sounds like the mixture of two instruments playing in unison or multiple speakers speaking at the same time, can still be sufficiently analyzed or separated using cues from the modulation domain. 
Large variety of contributions all focussed on the same idea

I brought an important aspect on the number of sources recently was not of much interest

- I started top-down approach, to understand the capabilities of state-of-the-art methods
- unison warping, unison commonfate, lead to Professionally produced music

Musical sources!?
% TODO: write Countnet, that it was extended to music scenarios
% TODO: Write in 06 that I tested redundancy reduction but it didn't work and cft is like JPG!
Counting audio sources can be transferred to counting bird species from their voice?

counting the leaves in a tree?

Generative Models, 
Combining COunting and Separation to build a super separation system.
Separation is static: Todays methods assume a fixed number of sources (except for Deep Clustering), are trained using balanced data. More flexible analysis of modulation could be: how many sources different sources are in the mix? separate all of them.

Datasets, data is everything. I created a Multimedia dataset. 
Separation turned into a machine learning task.

we started simple and extended to the more general case

Paradigma shift on audio signal processing almost all contributions today are in the field of deep learning and AI.
DL allowed to solve highly overlapped signals but not all scenarios
My thesis was in between this shift, but I was excited to try bold new ideas. 
Modulation aspects of signals are still important in the age of DL
Lots of open questions: how can long term relationship be modeled?
Source Counting needs to be applied for separation, but current models are not flexible enough (Deep Clustering?

Stepping back, our future is AI.
Optimal input representation for DNNs are not found yet. Raw waveform~\cite{Dieleman14, wavenet?} are promising but not very efficient, as the DNN needs to first learn a filterbank representation. Modulational features like common fate.

Counting is an extremely simple task, yet Machines struggle with it. 
Counting musical sources

  In the mean time\cite{stoeter18},  methods exist, that do perform comparably to the oracles methods (Ideal Binary Mask) for many tracks.
 it suggesting that the automatic karaoke problem can now be considered
solved to a large extent, given sufficient amounts of training data.


% TODO add reference to show equivalence between NMF/NTF and DNNs
- DNNs: Modern Architechtures, like CNNs and DNNs are more powerful but often do not significantly improve the results: SiSEC 2018. This is why domain knowledge can still be applied.

Find a way to make the representation more compact, having less redundancy but at the same time is still able to give good results.
we outperform state-of-the-art and stimulate n

- Combinations of DNNs and Factorization models are not done: ~\cite{vu16, leroux15}


In the mean time\cite{stoeter18}, methods exist, that do perform comparably to the oracles methods (Ideal Binary Mask) for many tracks it suggesting that the automatic karaoke problem can now be considered
solved to a large extent, given sufficient amounts of training data.